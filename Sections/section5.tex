\documentclass[]{article}
\usepackage{lmodern}
\usepackage{amssymb,amsmath}
\usepackage{ifxetex,ifluatex}
\usepackage{fixltx2e} % provides \textsubscript
\ifnum 0\ifxetex 1\fi\ifluatex 1\fi=0 % if pdftex
  \usepackage[T1]{fontenc}
  \usepackage[utf8]{inputenc}
  \usepackage{eurosym}
\else % if luatex or xelatex
  \ifxetex
    \usepackage{mathspec}
  \else
    \usepackage{fontspec}
  \fi
  \defaultfontfeatures{Ligatures=TeX,Scale=MatchLowercase}
  \newcommand{\euro}{€}
\fi
% use upquote if available, for straight quotes in verbatim environments
\IfFileExists{upquote.sty}{\usepackage{upquote}}{}
% use microtype if available
\IfFileExists{microtype.sty}{%
\usepackage{microtype}
\UseMicrotypeSet[protrusion]{basicmath} % disable protrusion for tt fonts
}{}
\usepackage[margin=1in]{geometry}
\usepackage{hyperref}
\hypersetup{unicode=true,
            pdftitle={Section 5: Manipulation et représentation de données : éviter des écueils classiques},
            pdfauthor={Datactivist, 2018-2019},
            pdfborder={0 0 0},
            breaklinks=true}
\urlstyle{same}  % don't use monospace font for urls
\usepackage{graphicx,grffile}
\makeatletter
\def\maxwidth{\ifdim\Gin@nat@width>\linewidth\linewidth\else\Gin@nat@width\fi}
\def\maxheight{\ifdim\Gin@nat@height>\textheight\textheight\else\Gin@nat@height\fi}
\makeatother
% Scale images if necessary, so that they will not overflow the page
% margins by default, and it is still possible to overwrite the defaults
% using explicit options in \includegraphics[width, height, ...]{}
\setkeys{Gin}{width=\maxwidth,height=\maxheight,keepaspectratio}
\IfFileExists{parskip.sty}{%
\usepackage{parskip}
}{% else
\setlength{\parindent}{0pt}
\setlength{\parskip}{6pt plus 2pt minus 1pt}
}
\setlength{\emergencystretch}{3em}  % prevent overfull lines
\providecommand{\tightlist}{%
  \setlength{\itemsep}{0pt}\setlength{\parskip}{0pt}}
\setcounter{secnumdepth}{0}
% Redefines (sub)paragraphs to behave more like sections
\ifx\paragraph\undefined\else
\let\oldparagraph\paragraph
\renewcommand{\paragraph}[1]{\oldparagraph{#1}\mbox{}}
\fi
\ifx\subparagraph\undefined\else
\let\oldsubparagraph\subparagraph
\renewcommand{\subparagraph}[1]{\oldsubparagraph{#1}\mbox{}}
\fi

%%% Use protect on footnotes to avoid problems with footnotes in titles
\let\rmarkdownfootnote\footnote%
\def\footnote{\protect\rmarkdownfootnote}

%%% Change title format to be more compact
\usepackage{titling}

% Create subtitle command for use in maketitle
\newcommand{\subtitle}[1]{
  \posttitle{
    \begin{center}\large#1\end{center}
    }
}

\setlength{\droptitle}{-2em}
  \title{Section 5: Manipulation et représentation de données : éviter des
écueils classiques}
  \pretitle{\vspace{\droptitle}\centering\huge}
  \posttitle{\par}
\subtitle{Culture générale des données}
  \author{Datactivist, 2018-2019}
  \preauthor{\centering\large\emph}
  \postauthor{\par}
  \date{}
  \predate{}\postdate{}


\begin{document}
\maketitle

layout: true

{Culture générale des données, Sciences Po Saint-Germain-en-Laye,
section 5}

\begin{center}\rule{0.5\linewidth}{\linethickness}\end{center}

class: center, middle

Ces slides en ligne : \url{http://datactivist.coop/SPoSGL/}

Sources : \url{https://github.com/datactivist/SPoSGL/}

Les productions de Datactivist sont librement réutilisables selon les
termes de la licence
\href{https://creativecommons.org/licenses/by-sa/4.0/legalcode.fr}{Creative
Commons 4.0 BY-SA}.

\subsection{Introduction}\label{introduction}

\paragraph{Ecoutez le .red{[}podcast introductif{]} de la section
5}\label{ecoutez-le-.redpodcast-introductif-de-la-section-5}

.center{[}{]}

.footnote{[}© Xavier Gorce{]}

\subsection{Médiane ou Moyenne ?}\label{madiane-ou-moyenne}

\begin{itemize}
\tightlist
\item
  Salaire mensuel \textbf{.red{[}moyen{]}} net en France en 2014
  (secteur privé) : \textbf{2 225€}
\item
  Salaire mensuel \textbf{.red{[}median{]}} net en France en 2014
  (secteur privé) : \textbf{1 783€} Soit une différence de près de
  450€ !
\end{itemize}

.center{[}\href{https://www.insee.fr/fr/statistiques/2121609\#consulter}{}{]}

\subsection{Corrélation ou causalité
?}\label{corralation-ou-causalita}

Rappel : une corrélation fortement positive, avec un coefficient de
corrélation (r) \textgreater{} 0,5, signifie seulement que deux
variables évoluent dans le même sens. Cela ne dit \textbf{rien} sur le
possible lien entre elles - Faites attention aux \textbf{corrélations
fallacieuses} ! - Exemple ici avec la corrélation quasi parfaite entre
la consommation de fromage par personne et le nombre de personnes qui
décèdent étranglées dans leurs draps

.center{[}\href{http://tylervigen.com/spurious-correlations}{}{]}

\subsection{Corrélation ou causalité
?}\label{corralation-ou-causalita-1}

\begin{itemize}
\tightlist
\item
  En ce qui concerne les corrélations, il faut être attentif au
  \textbf{.red{[}nombre d'unités observées{]}} : moins il y a
  d'observations, plus la corrélation est facilement élevée. Ce n'est
  pas pareil de regarder la corrélation entre 2 variables sur 13
  régions ou sur 35 000 villes
\end{itemize}

.center{[}{]}

\begin{itemize}
\tightlist
\item
  Il faut également être prudent sur l'échelle spatiale utilisée :
  la corrélation entre le \% d'immigrants et le vote FN est positif Ã~
  l'échelle du département. Mais Ã~ l'intérieur de ces départements,
  \textbf{ce sont dans les zones avec la plus faible proportion
  d'immigrants que les personnes votent FN}. Ainsi, au niveau micro la
  corrélation peut être inversée que ce qui était observé au niveau
  macro
\end{itemize}

\subsection{Corrélation ou causalité
?}\label{corralation-ou-causalita-2}

\begin{itemize}
\item
  Les zones combinant le plus de difficultés sont corrélées avec
  celles ayant une part des suffrages élevée pour Marine Le Pen Ã~ la
  présidentielle de 2012
\item
  Est-ce que cela signifie que les personnes qui ont le plus de
  difficultés votent FN ?
\item
  Pas forcément, pourquoi ?
\end{itemize}

\begin{quote}
Il serait cependant inexact dâ\euro{}™en déduire que ce vote est celui
des pauvres et des laissés pour compte. Ces derniers
sâ\euro{}™abstiennent le plus souvent. On doit plutôt constater que
câ\euro{}™est le vote des régions pauvres, celles où beaucoup
craignent les accidents de la vie car ils voient leurs proches atteints
par eux.
\end{quote}

Hervé Le Bras, dans son article pour The Conversation
\href{https://theconversation.com/la-france-inegale-qui-vote-fn-pas-forcement-ceux-a-qui-lon-pense-75977}{``La
France inégale : Qui vote FN ? Pas forcément ceux Ã~ qui lâ\euro{}™on
pense''} (\textbf{obligatoire})

\subsection{Les facteurs cachés}\label{les-facteurs-cachas}

Il existe des facteurs qu'on ne voit pas ou auxquels on ne pense
généralement pas mais qui peuvent fortement influer sur des
résultats. - Exemple ici avec la forte baisse du nombre de décès sur
les routes en France entre 2003 et 2014

.center{[}{]}

\begin{itemize}
\tightlist
\item
  Est-ce que les politiques actives de lutte contre l'insécurité
  routière mises en place dans les années 2000 sont la seule
  explication de cette baisse ? D'autres facteurs ``cachés'' peuvent
  aussi avoir eu un rôle significatif
\end{itemize}

\subsection{Le périmètre du
dénominateur}\label{le-parimatre-du-danominateur}

Lorsqu'il est question de \textbf{.red{[}ratio{]}}, il est utile de se
demander ce que comprend le dénominateur de ce ratio. Le périmètre du
dénominateur influence évidemment le résultat obtenu.

\begin{itemize}
\item
  Exemple avec le taux de chômage des jeunes en France : Ã~ la fin
  2017, le taux de chômage des 15-24 ans atteignait 23\%
\item
  Est-ce que cela signifie que près d'un jeune sur 4 est au chômage ?
\end{itemize}

.center{[}{]}

.footnote{[}\href{https://www.huffingtonpost.fr/2012/09/06/taux-chomage-bit-jeunes-dom_n_1860232.html}{Source}{]}

class: inverse, center, middle

\subsection{2. Les représentations graphiques et spatiales
problématiques}\label{les-reprasentations-graphiques-et-spatiales-problamatiques}

\subsection{Les représentations graphiques
problématiques}\label{les-reprasentations-graphiques-problamatiques}

Voici ce que donne le même graphique mais avec l'axe des ordonnées
démarrant Ã~ 0 :

.pull-left{[}{]}

.pull-right{[}\emph{Pas le même rendu, n'est-ce pas ? ;)}{]}

.pull-right{[}*NB: de nombreux exemples cités dans cette partie
proviennent de
\href{https://abonnes.lemonde.fr/les-decodeurs/article/2018/05/22/sept-conseils-pour-ne-pas-se-faire-avoir-par-les-representations-graphiques_5302680_4355770.html}{cet
article des Décodeurs*} du Monde{]}

\subsection{Les représentations graphiques
problématiques}\label{les-reprasentations-graphiques-problamatiques-1}

\begin{itemize}
\tightlist
\item
  Les \textbf{.red{[}graphiques en camembert{]}} (pie chart) sont Ã~
  éviter lorsqu'on souhaite représenter des \textbf{proportions} :
\end{itemize}

.center{[}\href{https://twitter.com/maxcroser/status/857389434756505600}{}{]}

\subsection{Les représentations graphiques
problématiques}\label{les-reprasentations-graphiques-problamatiques-2}

\begin{itemize}
\tightlist
\item
  Les \textbf{.red{[}graphiques avec une ``Base 100''{]}} : en fonction
  de l'année sélectionnée pour être la base 100, les évolutions
  peuvent fortement changer
\item
  A gauche, Production industrielle en Europe, base 100 = \textbf{2010}
\item
  A droite, Production industrielle en Europe, base 100 = \textbf{1974}
\end{itemize}

.pull-left{[}{]}

.pull-right{[}{]}

\subsection{Les représentations
spatiales}\label{les-reprasentations-spatiales}

En fonction de l'\textbf{unité spatiale} retenue pour représenter
graphiquement les données, les résultats peuvent différer fortement
d'une carte Ã~ une autre\ldots{} - Exemple avec la représentation des
résultats du 1er tour de l'élection présidentielle française de
2017. A gauche les résultats par commune, Ã~ droite une carte Ã~
anamorphose où la superficie des communes est proportionnelle au nombre
de leurs habitants

.center{[}\href{https://www.populationdata.net/cartes/france-elections-presidentielles-2017-1er-tour/}{}{]}

\subsection{Les représentations
spatiales}\label{les-reprasentations-spatiales-1}

Lorsqu'elles sont representées sur des cartes, les données doivent
souvent être représentées \textbf{relativement Ã~ la population}, et
non pas en valeur absolue - Exemple ici avec, Ã~ gauche, le nombre de
décès liés Ã~ l'alcoolisme en France en 2013. Et Ã~ droite, les
mêmes données représentées relativement Ã~ la population de chaque
région

.pull-left{[}{]}

.pull-right{[}{]}

.footnote{[}\href{https://www.scoresante.org/sindicateurs.html}{source}{]}

class: inverse, center, middle

\subsection{3. Les limites inhérentes aux
indicateurs}\label{les-limites-inharentes-aux-indicateurs}

\subsection{Les indicateurs ne sont pas
neutres}\label{les-indicateurs-ne-sont-pas-neutres}

.center{[}{]}

.footnote{[}\href{http://www.lefigaro.fr/conjoncture/2018/01/30/20002-20180130ARTFIG00342-le-trafic-de-drogue-va-bientot-entrer-dans-le-calcul-du-pib-francais.php}{Source}{]}

\subsection{Les indicateurs ne sont pas
neutres}\label{les-indicateurs-ne-sont-pas-neutres-1}

\begin{itemize}
\tightlist
\item
  Par ailleurs, qu'en est-il de tous les crimes ou délits commis qui
  n'aboutissent pas Ã~ un dépôt de plainte ?
\item
  Selon une
  \href{https://www.interieur.gouv.fr/content/download/104138/823089/file/IA17.pdf}{étude
  d'Interstats}, le service statistique du ministère de l'intérieur,
  \textbf{seulement une victime sur 12 effectue un signalement auprès
  des forces de sécurité.}
\end{itemize}

.center{[}{]}

\begin{itemize}
\tightlist
\item
  En conclusion, l'Etat 4001 reflète davantage les évolutions de
  l'activité des forces de l'ordre que de la criminalité Ã~ proprement
  parler
\end{itemize}

\subsection{La loi de Goodhart}\label{la-loi-de-goodhart}

La \textbf{.red{[}loi de Goodhart{]}} est un concept qui met en lumière
la difficulté de concevoir et mesurer des indicateurs fiables qui sont
associés Ã~ des enjeux politiques, financiers ou sociaux. Ainsi,
\textbf{lorsqu'une mesure devient un objectif, elle cesse d'être une
bonne mesure}

\begin{quote}
Tout indicateur statistique cesse dâ\euro{}™Ãªtre un indicateur
statistique fiable dès lors quâ\euro{}™il fait lâ\euro{}™objet
dâ\euro{}™enjeux car il devient sujet Ã~ des manipulations
\end{quote}

Charles Goodhart, économiste

\textbf{Connaissez-vous des exemples d'indicateurs où la loi de
Goodhart peut être observée ?}

Le taux de chômage ! On se concentre la plupart du temps sur \textbf{le
nombre d'inscrits en catégorie A} (et ce sont ces chiffres qui font les
gros titres de la presse généralement). Mais est-ce bien pertinent ?

\subsection{La loi de Goodhart}\label{la-loi-de-goodhart-1}

Exemple : le taux de chômage

\begin{itemize}
\tightlist
\item
  En fonction de la catégorie Ã~ laquelle on s'intéresse, le taux de
  chômage peut prendre des directions très différentes\ldots{} Ici
  focus \textbf{catégorie A/B/C}
\end{itemize}

.center{[}{]}

\subsection{La loi de Goodhart}\label{la-loi-de-goodhart-2}

Exemple : le taux de chômage

\begin{itemize}
\tightlist
\item
  En fonction de la catégorie Ã~ laquelle on s'intéresse, le taux de
  chômage peut prendre des directions très différentes\ldots{} Ici
  focus \textbf{catégorie A/B/C, 50 ans ou plus}
\end{itemize}

.center{[}{]}

\subsection{La loi de Goodhart}\label{la-loi-de-goodhart-3}

Exemple : le taux de chômage

.center{[}{]}

.footnote{[}Taux de chômage US entre 2009 et 2017{]}

\subsection{La loi de Goodhart}\label{la-loi-de-goodhart-4}

Exemple : le taux de chômage

\begin{itemize}
\tightlist
\item
  Au-delÃ~ du chômage, qu'en est-il du \textbf{.red{[}halo autour du
  chômage{]}} ? Comprend le travail occasionnel, le sous-emploi ou les
  personnes inactives
\end{itemize}

.center{[}{]}

Le taux de chômage (au sens du BIT) en France a plutôt eu tendance Ã~
baisser depuis 2015

class: inverse, center, middle

\subsection{Bibliographie}\label{bibliographie}

class: inverse, center, middle

\subsection{Quizz section 5 : rdv sur votre espace e-campus
!}\label{quizz-section-5-rdv-sur-votre-espace-e-campus}

\begin{center}\rule{0.5\linewidth}{\linethickness}\end{center}

class: inverse, center, middle

\section{Merci !}\label{merci}

Contact :
\href{mailto:timothee@datactivi.st}{\nolinkurl{timothee@datactivi.st}}


\end{document}
